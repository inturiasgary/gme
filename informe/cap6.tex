%\documentclass[12pt]{article}
%\usepackage[latin1]{inputenc}
%\usepackage[spanish]{babel}
%\usepackage{latexsym}
%\usepackage{amssymb}
%\usepackage{amsmath}

%\setlength{\textwidth}{15cm}

%\newtheorem{ejem}{Ejemplo}
%\newtheorem{teor}{Teorema}
%\begin{document}
\chapter{APÉNDICE}
\section{DOCUMENTACIÓN}
Como documentación base presentamos los manuales de usuario y manual técnico del Sistema Web microblog para notificación de acciones aplicadas en repositorios locales de desarrollo de proyectos de software versionados con GIT. 
\subsection{MANUAL DE USUARIO}
El presente documento pretende explicar la operativa a seguir para la correcta utilización del sistema Web microblog para notificación de acciones aplicadas en repositorios locales de desarrollo de proyectos de software versionados con GIT.
\subsubsection{INTRODUCCIÓN}
\subsubsection{ENTRADA AL SISTEMA}

\subsection{MANUAL TÉCNICO}
, tanto como la preparación del ambiente de trabajo básico tanto en el lado cliente como en el lado servidor.
\subsubsection{PREPARACIÓN DEL SISTEMA DE CONTROL DE VERSIONES}

En esta primera etapa de la documentación se detallará la preparación del ambiente de trabajo y la configuración respectivo de GIT, todos los procedimientos se dará énfasis en base al sistema operativo linux  'distribución UBUNTU', de forma similar los procedimientos se pueden realizar en otros sistemas basados en Mac OS X o Solaris, con algunas variaciones no muy significativas.

\paragraph{INSTALACIÓN DE GIT}

En este apartado, se facilitará al usuario una guía para la instalación del sistema de control de versiones GIT debido que en la actualidad no se encuentra instalado en ningún sistema operativo por defecto. Los pasos para instalar GIT, dependen mucho del sistema operativo que usted tiene, como un requisito previo se requiere tener conocimiento del uso de linea de comandos de sistema en sistemas linux de preferencia en las distribuciones basadas en Debian.

Git es ofertada como una colección de paquetes, donde cada paquete puede ser instalada independientemente según cada necesidad. El paquete principal se llama \textit{git-core}, la documentación esta disponible en \textit{git-doc}, y otros paquetes a considerar como ser:

\begin{itemize}
 \item \textit{git-arch, git-cvs, git-svn}
Para transferir proyectos desde otros sistemas de versiones como Arch, CVS, o Subversion a Git o viceversa,
instale uno o mas de estos paquetes.

\end{itemize}


\begin{proc}
	\begin{verbatim}
 		$ sudo aptitude install git-core
	\end{verbatim}
\end{proc}
\subsection{hola}
%\end{document}
