%\documentclass[12pt]{article}
%\usepackage[latin1]{inputenc}
%\usepackage[spanish]{babel}
%\usepackage{latexsym}
%\usepackage{amssymb}
%\usepackage{amsmath}

%\setlength{\textwidth}{15cm}

%\newtheorem{ejem}{Ejemplo}
%\newtheorem{teor}{Teorema}
%\begin{document}
\chapter{APÉNDICE}
\section{DOCUMENTACIÓN}
Como documentación base presentamos los manuales de usuario y manual técnico del Sistema Web microblog para notificación de acciones aplicadas en repositorios locales de desarrollo de proyectos de software versionados con GIT. 
\subsection{MANUAL DE USUARIO}
El presente documento pretende explicar la operativa a seguir para la correcta utilización del sistema Web microblog para notificación de acciones aplicadas en repositorios locales de desarrollo de proyectos de software versionados con GIT.
\subsubsection{INTRODUCCIÓN}
\subsubsection{ENTRADA AL SISTEMA}

\subsection{MANUAL TÉCNICO}
En este apartado explicaremos las etapas a seguir, tanto para poner en funcionamiento el sistema microblog con la instalación del comando de configuración en el lado cliente.
\subsubsection{PREPARACIÓN DEL SISTEMA DE CONTROL DE VERSIONES}

En esta primera etapa de la documentación se detallará la preparación del ambiente de trabajo y la configuración respectivo de GIT, todos los procedimientos se dará énfasis en base al sistema operativo Linux  'distribución UBUNTU versión actual 9.04', de forma similar los procedimientos se pueden realizar en otros sistemas basados en Mac OS X o Solaris, con algunas variaciones no muy significativas.
El sistema de control de versiones GIT tiene que estar presente y debidamente instaladas en todas las maquinas donde se desarrollaran los proyectos.
Durante la elaboración de este manual técnico, las instrucciones se elaboraron en base a la versión 1.6 de GIT, es recomendable realizar los procedimientos de instalación con la misma
versión o caso contrario con alguna mas actual, debido a que con versiones anteriores hubieron cambios significativos a la versión presente. 
\paragraph{INSTALACIÓN DE GIT}

En este apartado, se facilitará al usuario una guía para la instalación del sistema de control de versiones GIT debido que en la actualidad no se encuentra instalado en ningún sistema operativo por defecto. Los pasos para instalar GIT, dependen mucho del sistema operativo que usted tiene, como un requisito previo se requiere tener conocimiento del uso de linea de comandos de sistema en Linux de preferencia en las distribuciones basadas en Debian.
\subparagraph{Debian/Ubuntu}
Git es ofertada como una colección de paquetes, donde cada paquete puede ser instalado independientemente según cada necesidad. El paquete principal se llama \textit{git-core}, la documentación esta disponible en \textit{git-doc}, y otros paquetes a considerar como ser:

\begin{itemize}
 \item \textit{git-arch, git-cvs, git-svn}
Para transferir proyectos desde otros sistemas de versiones como Arch, CVS, o Subversion a Git o viceversa,
instale uno o mas de estos paquetes.

\item \textit{git-gui, gitk, gitweb}
git-gui es una interfaz gráfica de usuario basado en Tcl/Tk , gitk es otra similar herramienta también basado en Tcl/Tk pero en enfocado al historial del repositorio.
gitweb esta escrito en Perl y muestra el repositorio Git en un navegador web.

\item \textit{git-cmail}
Es un componente esencial si quieres enviar parches Git a través de correo electrónico, esto es una practica común en algunos proyectos.

\item \textit{git-dacmon-run}
Para compartir tu repositorio, instala este paquete. crea un servicio demonio que te permitirá compartir tu repositorio a través de peticiones anónimas de descargas.
\end{itemize}
Para la instalación en el sistema operativo UBUNTU,tenemos que ejecutar la siguiente instrucción en el interprete de comandos:
\begin{verbatim}
 $ sudo aptitude install git-core git-doc gitweb git-gui gitk git-email
\end{verbatim}

\subparagraph{Windows}
Hay dos paquetes Git competidores para Windows: Cygwin y msysGit, funcionan de igual forma a excepción que \textit{git-svn} no tiene soporte apropiado en msysGit,
así que si tu necesitas alguna interoperabilidad entre Git y Subversion deberías usar la versión Cygwin.

El paquete Git Cygwin se encuentra disponible en su pagina web \textit{http://cygwin.com}, hay que descargar y ejecutar \textit{setup.exe}, y configurar de preferencia
con las opciones por defecto.

El paquete msysGit es fácil de instalar en Windows ya que contiene todos las dependencias necesarias para su ejecución. msysGit a sido diseñado para una completa integración 
con el estilo nativo de aplicaciones Windows. Para su instalación hay que descargar la ultima versión del instalador de su pagina web \textit{http://code.google.com/p/msysgit}
	\begin{verbatim}
 		$ sudo aptitude install git-core
	\end{verbatim}
\paragraph{CONFIGURACIÓN DE GIT}
Una vez realizada la instalación de Git en las maquinas clientes vamos a realizar la configuración básica, vamos a hacer uso del nombre juan perez, para explicar la configuración de Git, haciendo enfoque en el sistema operativo UBUNTU.
la siguientes instrucciones deben ejecutarse en el interprete de comandos:

Configuración global de Git.
\begin{verbatim}
 $ git config --global user.name "Juan Perez"
 $ git config --gloabl user.email "juanperez@gmail.com"
\end{verbatim}
Configuracion personalizada en cada repositorio, esto crea el archivo \textit{config} dentro de nuestra carpeta \textit{.git} en nuestro repositorio actual, esta forma es
estrictamente recomendada para realizar la interaccion con el sistema Web Microblog ya que usuarios multiples podrian trabajar en el mismo computador.
\begin{verbatim}
 $ git config user.name "Juan Perez"
 $ git config user.email "juanperez@gmail.com"
\end{verbatim}
\paragraph{INSTALACIÓN DE LA LIBRERIA GITPYTHON}
GitPython es una libreria para el lenguaje de programacion Python que nos servira para interactuar con
los repositorios creados por Git.

para la instalacion se deben realizar los siguientes pasos:
realizar la siguiente instruccion en el interprete de linea de codigos
\begin{verbatim}
 $ easy_install GitPython
\end{verbatim}

Este comando bajara la version mas reciente de GitPython y lo instalara en el sistema.

Alternativamente se puede instalar desde la distribucion, el repositorio git de GitPython esta disponible en Gitorious, ubicado en:

http://gitorious.org/projects/git-python/

y podemos conseguir un clon del repo desde:

git://gitorious.org/git-python/mainline.git

Una ves adquirido el repositorio, ingresamos en el directorio donde esta ubicado y ejecutamos la siguiente instruccion:
\begin{verbatim}
 $ python setup.py install
\end{verbatim}

con esto tendremos instalado la libreria para que nuestro cliente pueda enviar informacion remota hacia el sistema web microblog.