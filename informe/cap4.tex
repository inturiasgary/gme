%\documentclass[12pt]{article}
%\usepackage[latin1]{inputenc}
%\usepackage[spanish]{babel}
%\usepackage{latexsym}
%\usepackage{amssymb}
%\usepackage{amsmath}

%\setlength{\textwidth}{15cm}

%\newtheorem{ejem}{Ejemplo}
%\newtheorem{teor}{Teorema}
%\begin{document}
\chapter{Desarrollo del proyecto}
El desarrollo del proyecto esta enfocado en dos ambientes, uno en el lado cliente, que se desarrollaron scripts para poder permitir la comunicación de los repositorios en GIT
el otro lado en el lado servidor para la elaboración del sistema web, que mantendrá informado a los desarrolladores
\section{Metodología de desarrollo}
El desarrollo del proyecto 'Desarrollar un sistema Web microblog para notificación de acciones aplicadas en repositorios locales de desarrollo de proyectos de software versionados con GIT' Esta basado en la metodología de desarrollo SCRUM, apoyado también por algunos diagramas UML.
Scrum es un proceso en el que se aplican de manera regular un conjunto de mejores prácticas para trabajar en equipo y obtener el mejor resultado posible de un proyecto. Estas prácticas se apoyan unas a otras y su selección tiene origen en un estudio de la manera de trabajar de equipos altamente productivos.

En Scrum se realizan entregas parciales y regulares del resultado final del proyecto, priorizadas por el beneficio que aportan al receptor del proyecto. Por ello, Scrum está especialmente indicado para proyectos en entornos complejos, donde se necesita obtener resultados pronto, donde los requisitos son cambiantes o poco definidos, donde la innovación, la competitividad y la productividad son fundamentales.

Scrum también se utiliza para resolver situaciones en que no se está entregando al cliente lo que necesita, cuando las entregas se alargan demasiado, los costes se disparan o la calidad no es aceptable, cuando se necesita capacidad de reacción ante la competencia, cuando la moral de los equipos es baja y la rotación alta, cuando es necesario identificar y solucionar ineficiencias sistemáticamente o cuando se quiere trabajar utilizando un proceso especializado en el desarrollo de producto.

El cual presenta muchos beneficios como ser:


%\end{document}
