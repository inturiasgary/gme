%Tamaño y tipo de documento que queremos escribir.
\documentclass[letterpaper, 12pt,double,graphicx,caption,rotating]{report}
\setcounter{secnumdepth}{4}
\setcounter{tocdepth}{4}
%Paquetes para el idioma, las tildes, etc.
\usepackage{longtable}
\usepackage[activeacute,spanish]{babel}
\usepackage[letterpaper]{geometry}
\usepackage[utf8]{inputenc}
\usepackage{multirow}
\usepackage{multicol}
\usepackage{anysize}

\usepackage[pdftex]{graphicx} % PDFLaTeX
\DeclareGraphicsExtensions{.png,.pdf,.jpg}

\marginsize{3.1cm}{2cm}{1.7cm}{1.5cm}
\renewcommand{\baselinestretch}{1.5}


\newtheorem{proc}{Procedimiento}

% Salto de línea tras título de secciones \paragraph
\makeatletter % necesario para que reconozca a '@' como carácter normal
\renewcommand{\paragraph}{\@startsection{paragraph}{4}{\z@}{-3.25ex \@plus
-1ex \@minus -.2ex}{1.5ex \@plus .2ex}{\normalfont\normalsize\bfseries}}
\makeatother % necesario para que restablezca '@' como carácter especial

%Abrimos el bloque para escribir el documento.
\begin{document}
%Generamos un indice de los contenidos
\title{\textbf{{\Large Sistema Web microblog para notificación de acciones aplicadas en repositorios locales de desarrollo de proyectos de software versionados con GIT.}\\[2cm]}}

\UNIVERSITY{\texorpdfstring{\href{University Web Site URL Here (include http://)}
                {UNIVERSIDAD MAYOR DE SAN SIMON}\\
		{FACULTAD DE CIENCIAS Y TECNOLOGÍA}\\
		{CARRERA: INGENIERÍA DE SISTEMAS}}}


\authors  {\texorpdfstring
            {\href{your web site or email address}{\textbf{Gary Inturias}}}
            {Gary Inturias}
          }
%{Tutor: Lic. Boris Marcelo Calancha Navia}
\addresses  {\groupname\\\deptname\\\univname}  % Do not change this here, instead these must be set in the "Thesis.cls" file, please look through it instead
\date       {Agosto, 2009}
\subject    {}
\keywords   {}
{
\begin{picture}(0,0)
\put(0,0){\includegraphics[height=2.5cm]{./imagenes/umss.jpg}}
\end{picture}
\begin{picture}(0,0)
\put(380,40){\includegraphics[height=2.5cm]{./imagenes/fcyt.jpg}}
\end{picture}}

\maketitle
\begin{flushright}
    {\Large \textbf{Dedicatoria}}\\[1cm]
    A mis padres y hermanos por\\ brindarme su apoyo incondicional. \\
\end{flushright}
\begin{center}
{\Large \textbf{Agradecimientos}}\\[2cm]
{\leftskip=8cm
A mis padres Juan Carlos Inturias y Celida Rojas por el apoyo incondicional que me dieron a lo largo de mi vida.\\
A mis hermanos Sonia y Ruddy por el apoyo que me dieron.\\
A la Universidad\\
Al Lic. Boris M. Calancha por sus conocimientos impartidos y las sugerencias realizadas a lo largo de mi enseñanza universitaria y el apoyo en la realización de este proyecto de grado.\\
A todas las personas que de algún modo sus criticas y sus enseñanza me ayudaron para poder enfrentar los retos de la vida.\\[3cm]
\par}
\textbf{\textit{¡Muchas Gracias!}}
\end{center}
\begin{titlepage}

\begin{center}
 {\Large FICHA RESUMEN}\\
\end{center}

El desarrollo de proyecto de software en la actualidad se encuentra a cargo de un equipo de desarrolladores, como así también los administradores del proyecto.
La comunicación entre los integrantes es esencial para el cumplimiento de los objetivos a alcanzar, para la coordinación eficaz del equipo de trabajo y la determinación de tiempos para las respectivas tareas designadas.\\

La dificultad de tratar con cada uno de los integrantes de forma personal aumenta según el numero que estos integran y aun mas si realizan sus trabajos a distancia, esto causa mala comprensión de las tareas a cumplir dentro del desarrollo de software. Provocando perdidas de tiempo que en el transcurso son esenciales para la culminación del proyecto.\\

La mayoría de sistemas de control de versiones no se encuentran integrados a un sistema que de a conocer las actividades que se realizan en cada uno de los repositorios que modifican cada integrante desarrollador, Así provocando una sincronización causando conflictos a la hora de realizar actualizaciones en los repositorios.\\
El sistema de control de versiones GIT no es una excepción a estos posibles problemas, debido a que cada integrante del grupo de desarrollo obtiene una copia idéntica del repositorio, el cual es una de las cualidades de los sistemas de control de versiones distribuidas.\\

Es por esto que surge la idea de aportar con una herramienta para facilitar las notificaciones de las acciones que se efectúan en los repositorios de trabajo, mediante un sistema web microblog y así poder reducir los conflictos de inconsistencias debido a las modificaciones de ficheros por varias personas no coordinadas, también se podrá registrar las listas de tareas asignadas a cada integrante 
perteneciente al repositorio en desarrollo.


\end{titlepage}
\newpage
\tableofcontents
\newpage
\listoftables
%\documentclass[12pt]{article}
%\usepackage[latin1]{inputenc}
%\usepackage[spanish]{babel}
%\usepackage{latexsym}
%\usepackage{amssymb}
%\usepackage{amsmath}

%\setlength{\textwidth}{15cm}

%\newtheorem{ejem}{Ejemplo}
%\newtheorem{teor}{Teorema}
%\begin{document}
\chapter{PRESENTACIÓN}
\section{INTRODUCCIÓN}
La dependencia causada por los sistemas informáticos en casi todas las actividades realizadas en los países, como ser la fabricación industrial, sistemas financieros, productos eléctricos, entre otros,
hacen que estos sistemas informáticos por la complejidad que presentan sean construidos a través de la ingeniería de software que comprenden de un conjunto de técnicas y herramientas que están en gran crecimiento debido a las exigencias, sin embargo mientras mas crezca nuestra capacidad de producir software, también lo hará la complejidad de los sistemas de software solicitados.\\

Es así como surgen diferentes herramientas para brindar apoyo dentro de las etapas de la ingeniera de software según cada metodología desempeñada. Una de las etapas que es imprescindible dentro de todas estas metodologías se encuentra el de implementación en el cual encontramos a una herramienta si bien no es obligatoria, resulta de gran ayuda e imprescindible hoy en día para la mayoría de las empresas y personas que se dedican a la elaboración de sistemas, hablamos de los sistemas de control de versiones que nos ayudan a la administración de las distintas versiones de cada producto desarrollado.\\

Encontramos una gran diversidad de sistemas de control de versiones, pero en este caso hacemos hincapié a uno de los mas populares como ser GIT, un sistema si bien no tan conocido como su adversario Subversion, pero que denota características ventajosas como ser la velocidad, reducción de tamaño,entre otras. es así como surge la idea de aportar una herramienta de apoyo para el desarrollo de software, como ser la comunicación de anuncios en tiempo real escritos por los integrantes del equipo de desarrollo como así también las acciones commit \footnote{commit (acción de cometer) se refiere a la idea de hacer que un conjunto de cambios "tentativos, o no permanentes" se conviertan en permanentes.} realizadas en cada uno de los repositorios locales que disponen, Publicadas a través de un sistema microblog.\\

pero ¿porque un microblog?, las actividades realizadas por cada integrante y las acciones en los repositorios son muy dinámicas por lo tanto sirven de gran ayuda el conocimiento del estado de cada uno de ellos, y lo hacemos mediante anuncios no muy largos publicados en el sistema de control de versiones, a través de forma automática gracias a la ayuda de algunos ganchos que nos brinda el sistema GIT y también a través del interprete de ordenes del sistema operativo, de esta manera se realizan las publicaciones de una manera menos perjuiciosa para el desarrollador, evitando perdidas de tiempo.\\

\section{PLANTEAMIENTO DEL PROBLEMA}

La comunicación entre los integrantes es esencial para el cumplimiento de los objetivos a alcanzar, para la coordinación eficaz del equipo de trabajo y la determinación de tiempos para las respectivas tareas asignadas.\\

La dificultad de tratar con cada uno de los integrantes de forma personal aumenta según el numero que estos integran y aun mas si realizan sus trabajos a distancia, esto causa mala comprensión de las tareas a cumplir dentro del desarrollo de software. Provocando perdidas de tiempo que en el transcurso son esenciales para la culminación del proyecto.\\

La mayoría de sistemas de control de versiones no se encuentran integrados a un sistema microblog, de ser así estos ayudarían a notificar que cambios acometidos que se están realizando en tiempo real a los ficheros de los repositorios de trabajo. Esta falta de comunicación provoca sincronización, causando conflictos a la hora de realizar actualizaciones en los repositorios.\\

GIT no es una excepción a estos posibles problemas, debido a que cada integrante del grupo de desarrollo obtiene una copia idéntica del repositorio y es responsable a mantenerlo en funcionamiento, es una de las cualidades de los sistemas de control de versiones distribuidas.\\

\section{OBJETIVOS}
\subsection{OBJETIVO GENERAL}
Desarrollar un sistema Web microblog\footnote{Microblog es un servicio que permite a sus usuarios enviar y publicar mensajes breves.
} para notificación de acciones aplicadas en repositorios locales de desarrollo de proyectos de software versionados con GIT\footnote{GIT software de sistema de control de versiones distribuido.\\
}.

\subsection{OBJETIVOS ESPECÍFICOS}
\begin{itemize}
\item Elegir herramientas que aporten al desarrollo del sistema
\item Analizar herramientas que faciliten la integración del sistema de control de versiones GIT con aplicaciones Web.
\item Realizar un asistente de configuración, para la comunicación de 	repositorios con el sistema Web(lado cliente).
\item Desarrollar el sistema web microblog, que permita registrar las acciones realizadas(commit) por los desarrolladores en los repositorios locales(lado servidor).
\end{itemize}

\section{JUSTIFICACIÓN}
El sistema web que se desea desarrollar ayudara  a los integrantes del grupo de trabajo encargados en la realización del proyecto, facilitando hacer un seguimiento de las tareas a realizar.\\

Los proyectos gestionados con el sistema de control de versiones GIT se podrán integrar con el sistema web microblog, debido a que se hará uso de algunos ganchos(hooks) que enlazarán con el sistema y así poder notificar a los integrantes del grupo de desarrollo en tiempo real sobre las acciones(commit) realizadas en cada uno de sus repositorios locales.\\

\section{METODOLOGÍA}
SCRUM es una metodología ágil, toma en cuenta la situación cambiante de los requisitos del cliente, así también minimiza la necesidad de la documentación.
Organizando el desarrollo de software de un manera evolutiva en los denominados sprints(periodos que duran de 15 a 30 días).

\section{CRONOGRAMA DEL DESARROLLO}
\begin{center}
\begin{longtable}{|p{2cm}|p{2cm}|p{3cm}|p{2cm}|p{2cm}|p{1.2cm}|}
\hline
Objetivo General & Objetivo Específico & Actividad & Método & Resultado & Tiempo\\
\hline
\multirow{14}{2cm}{Desarrollar un sistema web microblog para notificación de acciones aplicadas en repositorios locales de desarrollo de proyectos de software versionados con GIT.} & \multirow{2}{2cm}
{Elegir herramientas que aporten al desarrollo del software.} & {Investigar marcos de trabajo para la facilidad de elaboración de sistemas Web.} &\multirow{2}{2cm} {Búsqueda bibliográficas, entrevistas.} &\multirow{2}{2cm} {Se determinará las herramientas a ser utilizadas durante el transcurso del desarrollo del proyecto.}&\multirow{2}{2cm}{2}\\\cline{3-3}
& & {Investigar Herramientas CASE para apoyo al desarrollo del proyecto.}&&&\\\cline{2-6}

&\multirow{3}{2cm} {Analizar herramientas que faciliten la integración del sistema de control GIT con aplicaciones Web.}& {Hacer un análisis de ventajas y desventajas referentes a las opciones encontradas para el desarrollo}&\multirow{3}{2cm}{Búsquedas bibliográficas, Entrevistas}&\multirow{3}{2cm}{Se habrán realizado pruebas para la factibilidad de la comunicación entre las maquinas locales(Cliente) y el sistema web. Se tendrá montado el ambiente de trabajo para la realización del proyecto.}&\multirow{3}{2cm}{2}\\\cline{3-3}
& &{Realizar la instalación del ambiente de trabajo.}&&&\\\cline{3-3}
& &{Realizar pruebas de comunicación con una maquina local a un sistema web simple}&&&\\\cline{2-6}

& \multirow{3}{2cm}{Realizar un asistente de configuración, para la comunicación de repositorios con el sistema Web.}&{Hacer una revisión de código de herramientas ya desarrolladas previamente.}&\multirow{3}{2cm}{Búsqueda bibliográfica. Analistas de código de programas ya realizados previamente.}&\multirow{3}{2cm}{Se determinará la forma de configuración, la posibilidades de fallo presentadas, se elaborarán los scripts para la comunicación cliente-sistema web.}&\multirow{3}{2cm}{2}\\\cline{3-3}
& &{Realizar scripts que faciliten la configuración de la comunicación cliente-sistema Web.}&&&\\\cline{3-3}
& &{Realizar pruebas de configuración.}&&&\\\cline{2-6}

&\multirow{6}{2cm}{Desarrollar el sistema web microblog, que permita registrar las acciones realizadas(commit) por los desarrolladores en los repositorios locales.}&{- Inicio del Sprint, Realizar una lista de requerimientos para el desarrollo del sistema según iteración.}&\multirow{6}{2cm}{Búsqueda bibliográfica, entrevistas, Respetando normas de la metodología SCRUM, Se realizan scripts de pruebas(unitTest y clientTest)}&\multirow{6}{2cm}{Se elaborarán los prototipos según cada iteración, esta comprende un total de 4 iteraciones.}&\multirow{6}{2cm}{2}\\\cline{3-3}

& &{Realizar algunos diagramas UML de apoyo para el desarrollo del sistema}&&&\\\cline{3-3}
& &{Determinación del alcance de los objetivos a cumplir durante la iteración}&&&\\\cline{3-3}
& &{Inicio del desarrollo del sistema según iteración}&&&\\\cline{3-3}
& &{Pruebas de validación al prototipo}&&&\\\cline{3-3}
& &{Retrospectiva de las actividades realizadas en el sprint actual}&&&\\
\hline
\end{longtable}
\end{center}
\subsection{PLANIFICACIÓN POR FASES}

\subsection{CALENDARIO DEL PROYECTO}

\subsubsection{ACTIVIDADES}
\begin{figure}[htb]
  Diagrama de Gantt
  \centering
  \includegraphics[width=0.9\textwidth]{imagenes/Gantt.png}%ext=pdf,jpg,png
  \caption{Diagrama de Gantt que muestra el tiempo de duración para el desarrollo del proyecto.}
  \label{contexto:figura}
\end{figure}

%\end{document}

%\documentclass[12pt]{article}
%\usepackage[latin1]{inputenc}
%\usepackage[spanish]{babel}
%\usepackage{latexsym}
%\usepackage{amssymb}
%\usepackage{amsmath}

%\setlength{\textwidth}{15cm}

%\newtheorem{ejem}{Ejemplo}
%\newtheorem{teor}{Teorema}
%\begin{document}

\chapter{MARCO TEÓRICO}
\section{DEFINICIONES BÁSICAS}
\subsection{EL PROCESO DE DESARROLLO DE SOFTWARE}

El proceso de desarrollo de software tiene como propósito la producción eficaz y eficiente de un producto software que reúna los requisitos del cliente. Este proceso es intensamente intelectual, afectado por la creatividad y juicio de las personas involucradas. Aunque un proyecto de desarrollo de software es equiparable en muchos aspectos a cualquier otro proyecto de ingeniería, en el desarrollo de software hay una serie de desafíos adicionales, relativos esencialmente a la naturaleza del producto obtenido. \\

En aspectos generales sin enfocarnos al concepto de cada metodología utilizada para realizar esta labor, podemos ver que nos enfocamos en la evolución del código.\\

Ya sea que trabajemos en grupo o de forma individual, esta actividad se comienza con una hoja en blanco y una idea en la cabeza, que a medida que progresamos, va tomando forma.
Enfocandonos desde el punto de vista del código realizado durante este proceso, al principio arrancamos con un directorio vacío, en el cual vamos escribiendo, construyendo nuestro software
de forma incremental e iterativa, corrigiendo nuestros propios errores a medida que aparecen, y agregando cosas nuevas cuando se nos place.\\

Nuestro código va evolucionando, cambiando con iteraciones pequeñas que nosotros vamos introduciendo por diversos motivos como, correcciones o características nuevas en el software.
Estos cambios realizados en el código son de manera organizada y ordenada, debido a que corresponden a alguna construcción mental nuestra, porque cuando nosotros pensamos en implementar algo o corregir un bug\footnote{Un bug es un mal funcionamiento de un elemento de software: que un programa haga cosas no queridas, o que no haga las cosas que debería.}, lo tenemos en mente como una unidad, como un objetivo bastante independiente de como se representa en el código fuente; y nosotros pensamos en que cambios le tenemos que realizar para lograr ese objetivo propuesto.

\subsection{GRUPOS DE TRABAJO EN EL DESARROLLO DE SOFTWARE}

En la actividad de desarrollo de software con frecuencia los desarrolladores son organizados en grupo, independientemente de que metodología empleen. Surgiendo la suma necesidad de coordinar el trabajo, no solo por una cuestión natural sino también como mecanismo para optimizar los recursos.
Por eso, para desarrollar en grupo es imprescindible la buena comunicación y el entendimiento entre los pares. Esto implica que, en general los desarrollos se dan de forma coordinada (ya sea de manera horizontal o vertical, independientemente del mecanismo que se elija explícita o implícitamente para ello) al menos en un nivel social, las tareas se reparten y los cambios se discuten donde afectan al grupo para facilitar el trabajo.\\

Claro que el desarrollar en grupo, por más que uno se lleve maravillosamente bien con la gente involucrada, acarrea ciertas incomodidades que sí son más técnicas. Al haber más de una persona modificando el código fuente de forma simultanea, existe una complejidad, y nada menor, en lo que refiere al hecho de sincronizarlo y mantenerlo coherente entre todos los miembros del grupo.\\

También se dan en un grupo de trabajo relaciones asimétricas respecto del código, debido a que cada grupo tiene una forma y un flujo de trabajo particular, en el cual, por ejemplo, se pueden dar relaciones jerárquicas, revisión de código entre pares, subgrupos, etc.\\

Esto va a reflejarse en el código fuente con el surgimiento de una nueva necesidad, que va a ser la de la integración de múltiples trabajos individuales, la distribución del mismo en distintas maquinas y la coordinación para que todos puedan trabajar sobre la misma base de código.

\subsection{FRAMEWORK DE DESARROLLO}

El concepto framework se emplea un muchos ámbitos del desarrollo de sistemas
software, Podemos encontrar frameworks para el desarrollo de aplicaciones médicas, de visión por computador, para el desarrollo de juegos, y para cualquier ámbito que pueda ocurrírsenos.\\

En general, con el término framework, nos estamos refiriendo a una estructura
software compuesta de componentes personalizables e intercambiables para el
desarrollo de una aplicación. En otras palabras, un framework se puede considerar como
una aplicación genérica incompleta y configurable a la que podemos añadirle las últimas
piezas para construir una aplicación concreta.\\

Los objetivos principales que persigue un framework son: 
acelerar el proceso de desarrollo ya que son diseñados con el intento de facilitar el desarrollo de software, permitiendo a los diseñadores y programadores pasar más tiempo identificando requerimientos de software que tratando con los tediosos detalles de bajo nivel de proveer un sistema funcional.
reutilizar código ya existente y promover buenas prácticas de desarrollo como el uso de patrones.\\

Fuera de las aplicaciones en la informática, puede ser considerado como el conjunto de procesos y tecnologías usados para resolver un problema complejo. Es el esqueleto sobre el cual varios objetos son integrados para una solución dada.

\subsubsection{ARQUITECTURA MVC}

Dentro de este aspecto, podemos basarnos en el modelo MVC ya que debemos fragmentar nuestra programación, debido a que la mayoría de los frameworks conocidos adoptan por esta arquitectura o caso contrario similar. Tenemos que contemplar estos aspectos básicos en cuanto a la implementación de nuestro sistema.\\
\begin{figure}[ht]
\centering
\includegraphics[width=0.7\textwidth]{imagenes/mvc.jpg}%ext=pdf,jpg,png
\caption{Patrón de diseño MVC}
\label{contexto:figura}
\end{figure}

\begin{itemize} 

\item \textbf{Modelo} es el responsable de:\\

\begin{itemize}
 \item Acceder a la capa de almacenamiento de datos. Lo ideal es que el modelo sea independiente del sistema de almacenamiento.
 \item Define las reglas de negocio (la funcionalidad del sistema).
 \item Lleva un registro de las vistas y controladores del sistema.
 \item Si estamos ante un modelo activo, notificará a las vistas los cambios que en los datos pueda producir un agente externo.\\
\end{itemize}

\item \textbf{Vista} es el responsable de:\\

\begin{itemize}
 \item Recibir datos del modelo y los muestra al usuario.
 \item Tienen un registro de su controlador asociado.
 \item Pueden dar el servicio de Actualización, para que sea invocado por el controlador o por el modelo cuando es un modelo activo.\\
\end{itemize}


\item \textbf{Controlador:}\\

\begin{itemize}
 \item Recibir los eventos de entrada.
 \item Contiene reglas de gestión de eventos, estas acciones pueden suponer peticiones al modelo o a las vistas.\\
\end{itemize}

\end{itemize}



\subsubsection{FRAMEWORK WEB}
Un framework Web, por tanto, podemos definirlo como un conjunto de
componentes (por ejemplo clases en java y descriptores y archivos de configuración en
XML) que componen un diseño reutilizable que facilita y agiliza el desarrollo de
sistemas Web.\\



\subsection{SISTEMA DE CONTROL DE VERSIONES}

Una versión, revisión o edición de un producto, es el estado en el que se encuentra en un momento dado en su desarrollo o modificación. Se llama control de versiones a la gestión de los diversos cambios que se realizan sobre los elementos de algún producto o una configuración del mismo. Los sistemas de control de versiones facilitan la administración de las distintas versiones de cada producto desarrollado, así como las posibles especializaciones realizadas (por ejemplo, para algún cliente específico).\\

El control de versiones se realiza principalmente en la industria informática para controlar las distintas versiones del código fuente. Sin embargo, los mismos conceptos son aplicables a otros ámbitos como documentos, imágenes, sitios web, etcétera.\\

Aunque un sistema de control de versiones puede realizarse de forma manual, es muy aconsejable disponer de herramientas que faciliten esta gestión (CVS, Subversion, SourceSafe, ClearCase, Darcs, Bazaar, Plastic SCM, Git, Mercurial, etc.).\\

La facilidad que estos sistemas aportan son:
\begin{itemize}
\item permite a los programadores comunicar fácilmente su trabajo a otros.
\item le permite a un equipo compartir el código.
\item mantener versiones separadas de “producción” que están siempre deployables.
\item permite el desarrollo simultáneo de diferentes características en el mismo código base.
\item mantiene la pista de todas las versiones viejas de archivos.
\item previene que se sobrescriba trabajo.
\end{itemize}


\subsubsection{CLASIFICACIÓN}

\begin{itemize}
\item \textbf{CENTRALIZADOS :}\\Se basan en un repositorio único central con toda la información de cambios realizados el cual es accesible por todos los desarrolladores.\\
al haber un solo repositorio, este es el encargado del manejo de las ramas, quedando todos dentro de este. Están basados en una linea de tiempo, necesitan un servidor con funcionamiento de tiempo completo y con conexión permanente
configurado con algún sistema de autentificacion y permisos, por tal motivo la configuración suele ser mas compleja, según la necesidad. 
una de las desventajas mayores que representa es la dependencia de conexión con el servidor, ya que toda la información es manejada por el repositorio. 
Generalmente hay dos tipos el Lock-Modify-Unlock y los que usan el modelo Copy-Modify-Merge. el primero es el mas simple y limitado, y consiste en que cada vez que un usuario quiere editar un archivo, este se bloquea y por lo tanto no puede ser
modificado por nadie mas, hasta que el usuario lo desbloquee. El segundo modelo propone lo siguiente: se hace una copia del estado actual del repositorio, se modifica y se aplica el conjunto de cambios al repositorio central haciendo un merge. El cual es una forma de trabajo mas conveniente.
\item \textbf{DISTRIBUIDOS :}\\Carece de un punto central de desarrollo por tanto los repositorios están distribuidos y descentralizados en diferentes maquinas que pueden o no ser independientes entre si, y 
técnicamente no hay ninguno mas importante que otro. Esto trae bastante beneficios al desarrollar, dado que cada desarrollador tenga su repositorio propio sobre el cual trabaje de una forma independiente, y periódicamente se pongan en común los trabajos de todos en algún repositorio convenido a tal efecto.

\end{itemize}

\subsection{MICROBLOG}

Los blogs, como los conocemos actualmente, son espacios diseñados para que los navegantes de Internet puedan escribir libremente temas de cualquier índole, pueden ser aspectos de su vida diaria, pequeños post sobre sucesos extraños, denuncias, etcétera; la capacidad de expresar y escribir se limita al autor del blog.

El microblog puede considerarse como el hermano menor de los blogs; a diferencia de éstos, los caracteres son muy limitados, entre 100 y 150. 


\subsection{SERVICIOS WEB}

Existen múltiples definiciones sobre lo que son los Servicios Web, lo que muestra su complejidad a la hora de dar una adecuada definición que englobe todo lo que son e implican. Una posible sería hablar de ellos como un conjunto de aplicaciones o de tecnologías con capacidad para interoperar en la Web. Estas aplicaciones o tecnologías intercambian datos entre sí con el objetivo de ofrecer unos servicios. Los proveedores ofrecen sus servicios como procedimientos remotos y los usuarios solicitan un servicio llamando a estos procedimientos a través de la Web.\\

Estos servicios proporcionan mecanismos de comunicación estándares entre diferentes aplicaciones, que interactúan entre sí para presentar información dinámica al usuario. Para proporcionar interoperabilidad y extensibilidad entre estas aplicaciones, y que al mismo tiempo sea posible su combinación para realizar operaciones complejas, es necesaria una arquitectura de referencia estándar.\\


\subsection{WEB API}
Una API es una interfaz de programación de aplicaciones (del inglés API: Application Programming Interface). Es un conjunto de rutinas que provee acceso a funciones de un determinado software.

Son publicadas por los constructores de software para permitir acceso a características de bajo nivel o propietarias, detallando solamente la forma en que cada rutina debe ser llevada a cabo y la funcionalidad que brinda, sin otorgar información a cerca de como se lleva a cabo la tarea. Son utilizadas por los programadores para construir sus aplicaciones sin necesidad de volver a programar funciones ya hechas por otros, reutilizando código que se sabe que está probado y que funciona correctamente.

En la web las API's también son sinónimos de Web Service, las API's son publicadas por sitios para brindar la posibilidad de realizar alguna acción o acceder a alguna característica o contenido que el sitio provee. 

Una API puede ser:

\begin{itemize}
 \item \textbf{DEPENDIENTE DEL LENGUAJE:}\\ Que es disponible solo por un lenguaje de programación dado, usando la sintaxis y elementos de ese lenguaje para hacer conveniente el uso del API en esa petición.
 \item \textbf{INDEPENDIENTE DEL LENGUAJE:}\\ Que puede ser llamado por un gran numero de de lenguajes de programación. Esta es una característica de un servicio al estilo de la API que no está vinculada a un determinado proceso o sistema y está disponible como una llamada a procedimiento remoto(RPC Remote ).
\end{itemize}

\subsection{LINEA DE COMANDOS}

Un intérprete de órdenes, intérprete de línea de órdenes, intérprete de comandos, terminal, consola, shell o su acrónimo en inglés CLI (por Command Line Interface) es un programa informático que actúa como interfaz de usuario para comunicar al usuario con el sistema operativo mediante pantalla completa o ventana que espera órdenes escritas por el usuario en el teclado (ej. cd directorio), los interpreta y los entrega al sistema operativo para su ejecución. La respuesta del sistema operativo se muestra al usuario en la misma ventana. A continuación, el programa shell queda esperando más instrucciones. Se interactúa con la información de la manera más sencilla posible, sin gráficas, sólo el texto crudo.\\

Por extensión, también se llama intérprete de comandos a algunas interfaces de programas (mayores) que comunican al usuario con el software o al cliente de un servidor como, por ejemplo, bancos de datos (MySQL, Oracle) u otros programas (openSSL, FTP), etc.\\

Dada la importancia de esta herramienta, existe ya desde los comienzos de la computación. Existen, para diversos sistemas operativos, para diversos hardware, y con diferente funcionalidad. Suelen incorporar características tales como control de procesos, redirección de entrada/salida, listado y lectura de ficheros, protección, comunicaciones y un lenguaje de órdenes para escribir programas por lotes o (scripts o guiones).\\

Su contraparte es la interfaz gráfica de usuario que ofrece una estética mejorada a costa de mayor consumo de recursos computacionales, una mayor vulnerabilidad por complejidad y, en general, una reducción en la funcionalidad ofrecida.\\

\begin{figure}[htb]
\centering
\includegraphics[width=0.6\textwidth]{imagenes/lineaDeComandos.png}%ext=pdf,jpg,png
\caption{... esquema de elementos involucrados en una linea de ordenes ...}
\label{contexto:figura}
\end{figure}


%\end{document}

%\documentclass[12pt]{article}
%\usepackage[latin1]{inputenc}
%\usepackage[spanish]{babel}
%\usepackage{latexsym}
%\usepackage{amssymb}
%\usepackage{amsmath}

%\setlength{\textwidth}{15cm}

%\newtheorem{ejem}{Ejemplo}
%\newtheorem{teor}{Teorema}
%\begin{document}
\chapter{HERRAMIENTAS PARA EL DESARROLLO DE SOFTWARE}
\section{INTRODUCCION}


\section{LENGUAJE DE PROGRAMACION PYTHON}

\subsection{FRAMEWORK WEB DJANGO}
Django es un framework de desarrollo web de código abierto, escrito en Python, que cumple en cierta medida el paradigma del Modelo Vista Controlador. Fue desarrollado en origen para gestionar varias páginas orientadas a noticias de la World Company de Lawrence, Kansas, y fue liberada al público bajo una licencia BSD en julio de 2005. La versión estable (a julio de 2009) es la 1.1.

Otras características de Django son:
\begin{itemize}

\item Un mapeador objeto-relacional.
\item Aplicaciones "enchufables" que pueden instalarse en cualquier página gestionada con Django.
\item Una API de base de datos robusta.
\item Un sistema incorporado de "vistas genéricas" que ahorra tener que escribir la lógica de ciertas tareas comunes.
\item Un sistema extensible de plantillas basado en etiquetas, con herencia de plantillas.
\item Un despachador de URLs basado en expresiones regulares.
\item Un sistema "middleware" para desarrollar características adicionales; por ejemplo, la distribución principal de Django incluye componentes middleware que proporcionan cacheo, compresión de la salida, normalización de URLs, protección CSRF y soporte de sesiones.
\item Soporte de internacionalización, incluyendo traducciones incorporadas de la interfaz de administración.
\item Documentación incorporada accesible a través de la aplicación administrativa (incluyendo documentación generada automáticamente de los modelos y las librerías de plantillas añadidas por las aplicaciones).

\end{itemize}


\subsection{GIT}
GIT es un sistema de control de versiones distribuida aunque tambien se puede realizar un trabajo de forma centralizada, su corazón es una coleccion de herramientas simples que ejecutan un almacenaje y un directorio de la historia del arbol.
\subsubsection{MODELO DE OBJETOS GIT}
Toda la información necesitada para representar la historia de un proyecto es almacenada en archivos referenciados por 40 dígitos.también denominado ``Object Name''
\paragraph{TIPOS DE OBJETOS}
\begin{itemize}
 \item \textbf{Blob} Usado para almacenar datos de archivos, generalmente archivos.
 \item \textbf{Tree} Básicamente igual a un directorio, hace referencia a otras ramas, arboles o blobs.
 \item \textbf{Commit} Apunta a un árbol simple, timestamp, autor de los cambios del anterior commit realizado.
 \item \textbf{Tag} Forma especifica para marcar un Commit. como por ejemplo una versión ralease especifico.
\end{itemize}

\subsubsection{HOOKS(GANCHOS)}
No son copiados durante la ejecucion del comando \textit{git clone}.En otras palabras, los hooks configurados en nuestro repositorio privado no son propagados ni alteran el 
comportamiento del nuevo clon.
La clasificacion de los hooks son:
\begin{itemize}
 \item \textit{"pre" Hooks} se ejecuta antes de que la accion se complete. Este tipo de hooks es mayormente usado para \textbf{falta del libro}
 \item \textit{"post" Hooks} se ejecuta despues de que la accion se complete y puede ser usado para noficaciones o ejecutar un proceso adicional.
\end{itemize}

\subsection{XML RPC}
Es un protocolo de llamada a procedimiento remoto, que utiliza HTTP como el transporte y XML para la codificacion de datos. XML-RPC esta diseñado para ser lo mas sencillo posible, permitiendo al mismo tiempo las estructuras mas complejas de datos que deben ser transmitidos, procesados y devueltos.

\subsubsection{XML-RPC contra. Otros protocolos}

XML-RPC no es la única manera de hacer llamadas de procedimiento remotos. Otros protocolos populares incluyen CORBA, DCOM y el SOAP. Cada uno de estos protocolos tiene ventajas y desventajas.

\begin{itemize} 

\item \textbf{XML-RPC contra CORBA}

CORBA es un protocolo popular para la escritura distribuida, usos orientados a objetos. Se utiliza típicamente en usos de varias filas de empresas. Recientemente, también es adoptado por el proyecto del gnome para la comunicación de interaplicacion.

CORBA es muy apoyado por muchos vendedores y varios proyectos de software libre. CORBA trabaja bien con Java y C++, y está disponible para muchos otras idiomas. CORBA también proporciona una lengua excelente de la definición de interfaz (IDL), permitiendo que usted defina APIs legibles, orientados al objeto.

Desafortunadamente, CORBA es muy complejo. Tiene una curva de aprendizaje escarpada, requiere esfuerzo significativo para ejecutar, y requiere a clientes bastante sofisticados.

\item \textbf{XML-RPC contra DCOM}

DCOM es respuesta de Microsoft a CORBA. Es grande si usted está utilizando ya componentes de COM, y usted no necesita hablar con los sistemas que no son de Microsoft. entonces, no le ayudará mucho.

\item \textbf{XML-RPC contra SOAP}
\end{itemize}
SOAP es muy similar a XML-RPC. , Trabaja también formando llamadas de procedimiento sobre el protocolo HTTP con documentos de XML. Desafortunadamente, el SOAP parece sufrir de arrastramiento de la especificación.

SOAP fue creado originalmente como colaboración entre UserLand, DevelopMentor y Microsoft. El lanzamiento público inicial era básicamente XML-RPC con los namespaces \footnote{En programación, un espacio de nombres (del inglés namespace), en su acepción más simple, es un conjunto de nombres en el cual todos los nombres son únicos.} y nombres de elemento más largos. Desde entonces, sin embargo, SOAP se ha volcado un grupo de trabajo de W3C.

Desafortunadamente, el grupo de trabajo ha estado agregando una lista de características extrañas al SOAP. En fecha la escritura actual, el SOAP apoya esquemas de XML, enumeraciones, híbridos extraños de structs y de órdenes, y tipos de encargo. Al mismo tiempo, varios aspectos SOAP son puesta en práctica definida.
\textbf{poner la imagen de la secuencia de git commit}

%\end{document}

%\documentclass[12pt]{article}
%\usepackage[latin1]{inputenc}
%\usepackage[spanish]{babel}
%\usepackage{latexsym}
%\usepackage{amssymb}
%\usepackage{amsmath}

%\setlength{\textwidth}{15cm}

%\newtheorem{ejem}{Ejemplo}
%\newtheorem{teor}{Teorema}
%\begin{document}
\chapter{DESARROLLO DEL PROYECTO}
El desarrollo del proyecto esta enfocado en dos ambientes, uno en el lado cliente, que se desarrollaron scripts para poder permitir la comunicación de los repositorios en GIT
el otro lado en el lado servidor para la elaboración del sistema web, que mantendrá informado a los desarrolladores
\section{METODOLOGIA DE DESARROLLO}
El desarrollo del proyecto 'Desarrollar un sistema Web microblog para notificación de acciones aplicadas en repositorios locales de desarrollo de proyectos de software versionados con GIT' Esta basado en la metodología de desarrollo SCRUM.

Scrum es un proceso en el que se aplican de manera regular un conjunto de mejores prácticas para trabajar en equipo y obtener el mejor resultado posible de un proyecto. Estas prácticas se apoyan unas a otras y su selección tiene origen en un estudio de la manera de trabajar de equipos altamente productivos.

SCRUM es una metodología propuesta por los japoneses Hirotaka Takeuchi e Ikujijo Nonaka, un modo de desarrollo de carácter adaptable y orientada a las personas antes
que a los procesos, que controla de forma empírica y adaptable la evolución del proyecto con las siguientes prácticas de la gestión ágil:
       - Revisión de las iteraciones.
       - Desarrollo incremental, en el proyecto no se trabaja con diseño o
        abstracciones.
       - Desarrollo evolutivo, intenta predecir en las fases iniciales cómo será el
         resultado final y sobre dicha predicción realizar el diseño, la estructura del
        - producto no es realista, porque las circunstancias obligaran a remodelarlo
          muchas veces.
       - Auto-organización, los equipos son auto-organizados, con margen de decisión
         suficiente para tomar las decisiones que consideren oportunas.
       - Y colaboración entre todos según su conocimiento y no según su rol o puesto.

El resultado final en esta metodología se consigue de forma iterativa e incremental. Al comienzo de cada iteración (sprint) se determina qué partes se van a construir,
tomando como criterios la prioridad para el negocio, y la cantidad de trabajo que se podrá abordar durante la iteración. Dichas iteraciones se presentan en las etapas de
Especulación, Exploración y Revisión; y debido a que, según este tipo de modelos de desarrollo nunca se termina un producto, se presentan de forma infinita pudiendo
llegar a la etapa de cierre solo cuando se desee enviar al mercado una versión funcional del producto.

Reglas básicas
SCRUM tiene un conjunto de reglas muy pequeño y muy simple y está basado en los principios de inspección continua, adaptación, auto-gestión e innovación.
El cliente se entusiasma y se compromete con el proyecto dado que ve crecer el producto iteración a iteración y encuentra las herramientas para alinear el desarrollo
con los objetivos de negocio de su empresa.
Por otro lado, los desarrolladores encuentran un ámbito propicio para desarrollar sus capacidades profesionales y esto resulta en un incremento en la motivación de los
integrantes del equipo.
Ya hemos dicho que SCRUM es fácil y sencillo pero vamos a la materia, los siguientes son los elementos básicos de SCRUM:
    
    •   Una lista con las funcionalidades de la aplicación ordenadas de mayor a menor
        importancia. Esta lista se llama "Product Backlog". No hace falta que esta lista
        contenga todas las funcionalidades inicialmente.
    •   De la lista anterior, se toman las primeras funcionalidades, se descomponen en
        tareas y son anotadas en una lista que se llama "Sprint Backlog". Estas tareas
        serán realizadas en el siguiente mes.

Además de estos elementos tenemos unas cuantas reglas básicas y sencillas que
tenemos que cumplir:
    •   Una vez que se pasan las tareas más prioritarias del "Product Backlog" al
        "Sprint Backlog", estas no se pueden cambiar, esto quiere decir, que el trabajo
        de un mes queda fijado. Esta es la regla más importante de todas.
    •   Al final del mes, periodo denominado "Sprint", se tiene que tener un ejecutable
        con las funcionalidades del "Sprint Backlog".
    •   Todo el equipo puede añadir funcionalidades al "Product Backlog", pero sólo
        una persona puede ordenarlo. A esta persona se le denomina "Product
        Owner". Es el responsable del producto final.
    •   Cada día se hace una reunión de menos de 15 minutos, en la que se reúne
        todo el equipo: ingenieros y gestor (llamado "SCRUM Master") en la que cada
        miembro del equipo expone sólo los siguientes temas:
             o ¿Qué es lo que se hizo el día anterior?
             o ¿Qué es lo que se va a hacer hoy?
             o ¿Qué impedimentos tengo para realizar mi trabajo?


        Sólo se tratan estos temas para que la reunión sea rápida y no malgastar el
        tiempo de los demás. Si se tiene que tratar otro tema se hace otra reunión sólo
        con las personas implicadas. ¿Recordáis la serie de TV "Canción triste de Hill
        Street" en la que el sargento tenía una reunión matutina con sus agentes y que
        terminaba con un "Tengan cuidado ahí fuera"? pues viene a ser algo parecido.
    •   Al final del mes, es decir, al final del Sprint, se presenta el producto y se toman
        del "Product Backlog" las funcionalidades para cubrir en el siguiente mes.

La calidad en este caso, se logra y se mantiene de forma continua, pues se involucra
al cliente durante todo el tiempo que tarde el desarrollo, permitiéndole hacer aportes
que enriquezcan y generen nuevas funcionalidades y/o características al producto que
se está desarrollando.
Básicamente esto es todo.
Como se puede observar las reglas son sencillas, claras y es muy fácil de explicar y de
entender, lo que ayuda mucho a su implantación, pero no hay que engañarse, SCRUM
es un proceso de cambio, y uno además bastante serio.
SCRUM es sencillo, pero duro como una piedra, y se encuentra siempre mucha
resistencia:
    •   Los jefes de proyecto no querrán comenzar los proyectos sin tenerlo todo
        perfectamente identificado, acotado y planificado.
    •   Los desarrolladores no querrán la               responsabilidad    de  estimar   las
        funcionalidades y demostrar el producto.
    •   Los gerentes no querrán dejar tranquilo al equipo durante los Sprints.
    •   Etc.

\section{DESCRIPCION DEL PROBLEMA}

%\end{document}

%\documentclass[12pt]{article}
%\usepackage[latin1]{inputenc}
%\usepackage[spanish]{babel}
%\usepackage{latexsym}
%\usepackage{amssymb}
%\usepackage{amsmath}

%\setlength{\textwidth}{15cm}

%\newtheorem{ejem}{Ejemplo}
%\newtheorem{teor}{Teorema}
%\begin{document}
\chapter{CONCLUSIONES Y RECOMENDACIONES}
\section{CONCLUSIONES}
\section{RECOMENDACIONES}

%\end{document}

%\documentclass[12pt]{article}
%\usepackage[latin1]{inputenc}
%\usepackage[spanish]{babel}
%\usepackage{latexsym}
%\usepackage{amssymb}
%\usepackage{amsmath}

%\setlength{\textwidth}{15cm}

%\newtheorem{ejem}{Ejemplo}
%\newtheorem{teor}{Teorema}
%\begin{document}
\chapter{APÉNDICE}
\section{DOCUMENTACIÓN}
Como documentación base presentamos los manuales de usuario y manual técnico del Sistema Web microblog para notificación de acciones aplicadas en repositorios locales de desarrollo de proyectos de software versionados con GIT. 
\subsection{MANUAL DE USUARIO}
El presente documento pretende explicar la operativa a seguir para la correcta utilización del sistema Web microblog para notificación de acciones aplicadas en repositorios locales de desarrollo de proyectos de software versionados con GIT.
\subsubsection{INTRODUCCIÓN}
\subsubsection{ENTRADA AL SISTEMA}

\subsection{MANUAL TÉCNICO}
, tanto como la preparación del ambiente de trabajo básico tanto en el lado cliente como en el lado servidor.
\subsubsection{PREPARACIÓN DEL SISTEMA DE CONTROL DE VERSIONES}

En esta primera etapa de la documentación se detallará la preparación del ambiente de trabajo y la configuración respectivo de GIT, todos los procedimientos se dará énfasis en base al sistema operativo linux  'distribución UBUNTU', de forma similar los procedimientos se pueden realizar en otros sistemas basados en Mac OS X o Solaris, con algunas variaciones no muy significativas.

\paragraph{INSTALACIÓN DE GIT}

En este apartado, se facilitará al usuario una guía para la instalación del sistema de control de versiones GIT debido que en la actualidad no se encuentra instalado en ningún sistema operativo por defecto. Los pasos para instalar GIT, dependen mucho del sistema operativo que usted tiene, como un requisito previo se requiere tener conocimiento del uso de linea de comandos de sistema en sistemas linux de preferencia en las distribuciones basadas en Debian.

Git es ofertada como una colección de paquetes, donde cada paquete puede ser instalada independientemente según cada necesidad. El paquete principal se llama \textit{git-core}, la documentación esta disponible en \textit{git-doc}, y otros paquetes a considerar como ser:

\begin{itemize}
 \item \textit{git-arch, git-cvs, git-svn}
Para transferir proyectos desde otros sistemas de versiones como Arch, CVS, o Subversion a Git o viceversa,
instale uno o mas de estos paquetes.

\end{itemize}


\begin{proc}
	\begin{verbatim}
 		$ sudo aptitude install git-core
	\end{verbatim}
\end{proc}
\subsection{hola}
%\end{document}


%adicionamos la bibliografia
\begin{thebibliography}{2007}
\bibitem{}{LOELIGER, JON.Version control with Git, Ed. O'REILLY, 2009}
\bib
\bibitem{Orelly}Learning Python, 3a edition.
\bibitem{Wrox}Professional Python Framework Web 2.0 Programing with Django and Turbogears,2007.
\bibitem{Springer}Python Scripting for Computational Science, 3rd. edititon(2008). 
\bibitem{Pressman}Ingenieria de Sfotware,Roger Pressman
\bibitem{Scrum}Metodología agil SCRUM, http://www.proyectosagiles.org/que-es-scrum
\end{thebibliography}
\end{document}
%Cerramos el bloque para escribir el documento.
