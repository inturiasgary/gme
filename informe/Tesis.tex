%Tamaño y tipo de documento que queremos escribir.
\documentclass[letterpaper, 12pt,double,graphicx,caption,rotating]{book}

%Paquetes para el idioma, las tildes, etc.
\usepackage{longtable}
\usepackage[activeacute,spanish]{babel}
\usepackage[letterpaper,left=3cm,right=2.0cm]{geometry}
\usepackage[utf8]{inputenc}
\usepackage{multirow}
\usepackage{multicol}
\usepackage{parskip}
\renewcommand{\baselinestretch}{1.5}

\parskip  0.5cm
%Abrimos el bloque para escribir el documento.
\begin{document}
\begin{titlepage}
\begin{center}


UNIVERSIDAD MAYOR DE SAN SIMÓN\\
FACULTAD DE CIENCIAS Y TECNOLOGÍA\\
CARRERA: INGENIERÍA DE SISTEMAS\\[4cm]
 {\Large Sistema Web microblog para notificación de acciones aplicadas en repositorios locales de desarrollo de proyectos de software versionados con GIT.}\\[4cm]
 {\Large Proyecto de Grado, Presentado Para Optar al Diploma Académico de Licenciatura de Ingeniería de Sistemas}\\[3cm]

 {\Large Presentado por: Gary Inturias Rojas}\\[1cm]
 {\Large Tutor: Lic. Boris Marcelo Calancha Navia }\\[2cm]
 COCHABAMBA - BOLIVIA\\
 Agosto, 2009
\end{center}
\newpage
\vfill\par\noindent
\begin{flushright}
    {\Large \textbf{Dedicatoria}}\\[1cm]
    A mis padres y hermanos por\\ brindarme su apoyo incondicional. \\
\end{flushright}
\newpage


\begin{center}
{\Large \textbf{Agradecimientos}}\\[2cm]
{\leftskip=8cm
A mis padres Juan Carlos Inturias y Celida Rojas por el apoyo incondicional que me dieron a lo largo de mi vida.\\
A mis hermanos Sonia y Ruddy por el apoyo que me dieron.\\
A la Universidad\\
Al Lic. Boris M. Calancha por sus conocimientos impartidos y las sugerencias realizadas a lo largo de mi enseñanza universitaria y el apoyo en la realización de este proyecto de grado.\\
A todas las personas que de algún modo sus criticas y sus enseñanza me ayudaron para poder enfrentar los retos de la vida.\\[7cm]
\par}
\textbf{\textit{¡Muchas Gracias!}}



\end{center}

\newpage
\begin{center}
 {\Large FICHA RESUMEN}\\
\end{center}\noindent
\baselinestretch{2.0} 
El desarrollo de proyecto de software en la actualidad se encuentra a cargo de un equipo de desarrolladores, como así también los administradores del proyecto.
La comunicación entre los integrantes es esencial para el cumplimiento de los objetivos a alcanzar, para la coordinación eficaz del equipo de trabajo y la determinación de tiempos para las respectivas tareas designadas.
La dificultad de tratar con cada uno de los integrantes de forma personal aumenta según el numero que estos integran y aun mas si realizan sus trabajos a distancia, esto causa mala comprensión de las tareas a cumplir dentro del desarrollo de software. Provocando perdidas de tiempo que en el transcurso son esenciales para la culminación del proyecto.
La mayoría de sistemas de control de versiones no se encuentran integrados a un sistema microblog, el cual facilitaría las notificaciones de las acciones que efectúan en tiempo real a los ficheros de los repositorios de trabajo. Así provocando una de-sincronización causando conflictos a la hora de realizar actualizaciones en los repositorios.
GIT no es una excepción a estos posibles problemas, debido a que cada integrante del grupo de desarrollo obtiene una copia idéntica del repositorio, es una de las cualidades de los sistemas de control de versiones distribuidas.

\newpage
\end{titlepage}

%Generamos un indice de los contenidos
\tableofcontents
\newpage\noindent

%\documentclass[12pt]{article}
%\usepackage[latin1]{inputenc}
%\usepackage[spanish]{babel}
%\usepackage{latexsym}
%\usepackage{amssymb}
%\usepackage{amsmath}

%\setlength{\textwidth}{15cm}

%\newtheorem{ejem}{Ejemplo}
%\newtheorem{teor}{Teorema}
%\begin{document}
\chapter{PRESENTACIÓN}
\section{INTRODUCCIÓN}
La dependencia causada por los sistemas informáticos en casi todas las actividades realizadas en los países, como ser la fabricación industrial, sistemas financieros, productos eléctricos, entre otros,
hacen que estos sistemas informáticos por la complejidad que presentan sean construidos a través de la ingeniería de software que comprenden de un conjunto de técnicas y herramientas que están en gran crecimiento debido a las exigencias, sin embargo mientras mas crezca nuestra capacidad de producir software, también lo hará la complejidad de los sistemas de software solicitados.\\

Es así como surgen diferentes herramientas para brindar apoyo dentro de las etapas de la ingeniera de software según cada metodología desempeñada. Una de las etapas que es imprescindible dentro de todas estas metodologías se encuentra el de implementación en el cual encontramos a una herramienta si bien no es obligatoria, resulta de gran ayuda e imprescindible hoy en día para la mayoría de las empresas y personas que se dedican a la elaboración de sistemas, hablamos de los sistemas de control de versiones que nos ayudan a la administración de las distintas versiones de cada producto desarrollado.\\

Encontramos una gran diversidad de sistemas de control de versiones, pero en este caso hacemos hincapié a uno de los mas populares como ser GIT, un sistema si bien no tan conocido como su adversario Subversion, pero que denota características ventajosas como ser la velocidad, reducción de tamaño,entre otras. es así como surge la idea de aportar una herramienta de apoyo para el desarrollo de software, como ser la comunicación de anuncios en tiempo real escritos por los integrantes del equipo de desarrollo como así también las acciones commit \footnote{commit (acción de cometer) se refiere a la idea de hacer que un conjunto de cambios "tentativos, o no permanentes" se conviertan en permanentes.} realizadas en cada uno de los repositorios locales que disponen, Publicadas a través de un sistema microblog.\\

pero ¿porque un microblog?, las actividades realizadas por cada integrante y las acciones en los repositorios son muy dinámicas por lo tanto sirven de gran ayuda el conocimiento del estado de cada uno de ellos, y lo hacemos mediante anuncios no muy largos publicados en el sistema de control de versiones, a través de forma automática gracias a la ayuda de algunos ganchos que nos brinda el sistema GIT y también a través del interprete de ordenes del sistema operativo, de esta manera se realizan las publicaciones de una manera menos perjuiciosa para el desarrollador, evitando perdidas de tiempo.\\

\section{PLANTEAMIENTO DEL PROBLEMA}

La comunicación entre los integrantes es esencial para el cumplimiento de los objetivos a alcanzar, para la coordinación eficaz del equipo de trabajo y la determinación de tiempos para las respectivas tareas asignadas.\\

La dificultad de tratar con cada uno de los integrantes de forma personal aumenta según el numero que estos integran y aun mas si realizan sus trabajos a distancia, esto causa mala comprensión de las tareas a cumplir dentro del desarrollo de software. Provocando perdidas de tiempo que en el transcurso son esenciales para la culminación del proyecto.\\

La mayoría de sistemas de control de versiones no se encuentran integrados a un sistema microblog, de ser así estos ayudarían a notificar que cambios acometidos que se están realizando en tiempo real a los ficheros de los repositorios de trabajo. Esta falta de comunicación provoca sincronización, causando conflictos a la hora de realizar actualizaciones en los repositorios.\\

GIT no es una excepción a estos posibles problemas, debido a que cada integrante del grupo de desarrollo obtiene una copia idéntica del repositorio y es responsable a mantenerlo en funcionamiento, es una de las cualidades de los sistemas de control de versiones distribuidas.\\

\section{OBJETIVOS}
\subsection{OBJETIVO GENERAL}
Desarrollar un sistema Web microblog\footnote{Microblog es un servicio que permite a sus usuarios enviar y publicar mensajes breves.
} para notificación de acciones aplicadas en repositorios locales de desarrollo de proyectos de software versionados con GIT\footnote{GIT software de sistema de control de versiones distribuido.\\
}.

\subsection{OBJETIVOS ESPECÍFICOS}
\begin{itemize}
\item Elegir herramientas que aporten al desarrollo del sistema
\item Analizar herramientas que faciliten la integración del sistema de control de versiones GIT con aplicaciones Web.
\item Realizar un asistente de configuración, para la comunicación de 	repositorios con el sistema Web(lado cliente).
\item Desarrollar el sistema web microblog, que permita registrar las acciones realizadas(commit) por los desarrolladores en los repositorios locales(lado servidor).
\end{itemize}

\section{JUSTIFICACIÓN}
El sistema web que se desea desarrollar ayudara  a los integrantes del grupo de trabajo encargados en la realización del proyecto, facilitando hacer un seguimiento de las tareas a realizar.\\

Los proyectos gestionados con el sistema de control de versiones GIT se podrán integrar con el sistema web microblog, debido a que se hará uso de algunos ganchos(hooks) que enlazarán con el sistema y así poder notificar a los integrantes del grupo de desarrollo en tiempo real sobre las acciones(commit) realizadas en cada uno de sus repositorios locales.\\

\section{METODOLOGÍA}
SCRUM es una metodología ágil, toma en cuenta la situación cambiante de los requisitos del cliente, así también minimiza la necesidad de la documentación.
Organizando el desarrollo de software de un manera evolutiva en los denominados sprints(periodos que duran de 15 a 30 días).

\section{CRONOGRAMA DEL DESARROLLO}
\begin{center}
\begin{longtable}{|p{2cm}|p{2cm}|p{3cm}|p{2cm}|p{2cm}|p{1.2cm}|}
\hline
Objetivo General & Objetivo Específico & Actividad & Método & Resultado & Tiempo\\
\hline
\multirow{14}{2cm}{Desarrollar un sistema web microblog para notificación de acciones aplicadas en repositorios locales de desarrollo de proyectos de software versionados con GIT.} & \multirow{2}{2cm}
{Elegir herramientas que aporten al desarrollo del software.} & {Investigar marcos de trabajo para la facilidad de elaboración de sistemas Web.} &\multirow{2}{2cm} {Búsqueda bibliográficas, entrevistas.} &\multirow{2}{2cm} {Se determinará las herramientas a ser utilizadas durante el transcurso del desarrollo del proyecto.}&\multirow{2}{2cm}{2}\\\cline{3-3}
& & {Investigar Herramientas CASE para apoyo al desarrollo del proyecto.}&&&\\\cline{2-6}

&\multirow{3}{2cm} {Analizar herramientas que faciliten la integración del sistema de control GIT con aplicaciones Web.}& {Hacer un análisis de ventajas y desventajas referentes a las opciones encontradas para el desarrollo}&\multirow{3}{2cm}{Búsquedas bibliográficas, Entrevistas}&\multirow{3}{2cm}{Se habrán realizado pruebas para la factibilidad de la comunicación entre las maquinas locales(Cliente) y el sistema web. Se tendrá montado el ambiente de trabajo para la realización del proyecto.}&\multirow{3}{2cm}{2}\\\cline{3-3}
& &{Realizar la instalación del ambiente de trabajo.}&&&\\\cline{3-3}
& &{Realizar pruebas de comunicación con una maquina local a un sistema web simple}&&&\\\cline{2-6}

& \multirow{3}{2cm}{Realizar un asistente de configuración, para la comunicación de repositorios con el sistema Web.}&{Hacer una revisión de código de herramientas ya desarrolladas previamente.}&\multirow{3}{2cm}{Búsqueda bibliográfica. Analistas de código de programas ya realizados previamente.}&\multirow{3}{2cm}{Se determinará la forma de configuración, la posibilidades de fallo presentadas, se elaborarán los scripts para la comunicación cliente-sistema web.}&\multirow{3}{2cm}{2}\\\cline{3-3}
& &{Realizar scripts que faciliten la configuración de la comunicación cliente-sistema Web.}&&&\\\cline{3-3}
& &{Realizar pruebas de configuración.}&&&\\\cline{2-6}

&\multirow{6}{2cm}{Desarrollar el sistema web microblog, que permita registrar las acciones realizadas(commit) por los desarrolladores en los repositorios locales.}&{- Inicio del Sprint, Realizar una lista de requerimientos para el desarrollo del sistema según iteración.}&\multirow{6}{2cm}{Búsqueda bibliográfica, entrevistas, Respetando normas de la metodología SCRUM, Se realizan scripts de pruebas(unitTest y clientTest)}&\multirow{6}{2cm}{Se elaborarán los prototipos según cada iteración, esta comprende un total de 4 iteraciones.}&\multirow{6}{2cm}{2}\\\cline{3-3}

& &{Realizar algunos diagramas UML de apoyo para el desarrollo del sistema}&&&\\\cline{3-3}
& &{Determinación del alcance de los objetivos a cumplir durante la iteración}&&&\\\cline{3-3}
& &{Inicio del desarrollo del sistema según iteración}&&&\\\cline{3-3}
& &{Pruebas de validación al prototipo}&&&\\\cline{3-3}
& &{Retrospectiva de las actividades realizadas en el sprint actual}&&&\\
\hline
\end{longtable}
\end{center}
\subsection{PLANIFICACIÓN POR FASES}

\subsection{CALENDARIO DEL PROYECTO}

\subsubsection{ACTIVIDADES}
\begin{figure}[htb]
  Diagrama de Gantt
  \centering
  \includegraphics[width=0.9\textwidth]{imagenes/Gantt.png}%ext=pdf,jpg,png
  \caption{Diagrama de Gantt que muestra el tiempo de duración para el desarrollo del proyecto.}
  \label{contexto:figura}
\end{figure}

%\end{document}

%\documentclass[12pt]{article}
%\usepackage[latin1]{inputenc}
%\usepackage[spanish]{babel}
%\usepackage{latexsym}
%\usepackage{amssymb}
%\usepackage{amsmath}

%\setlength{\textwidth}{15cm}

%\newtheorem{ejem}{Ejemplo}
%\newtheorem{teor}{Teorema}
%\begin{document}
\chapter{DESARROLLO DEL PROYECTO}
El desarrollo del proyecto esta enfocado en dos ambientes, uno en el lado cliente, que se desarrollaron scripts para poder permitir la comunicación de los repositorios en GIT
el otro lado en el lado servidor para la elaboración del sistema web, que mantendrá informado a los desarrolladores
\section{METODOLOGIA DE DESARROLLO}
El desarrollo del proyecto 'Desarrollar un sistema Web microblog para notificación de acciones aplicadas en repositorios locales de desarrollo de proyectos de software versionados con GIT' Esta basado en la metodología de desarrollo SCRUM.

Scrum es un proceso en el que se aplican de manera regular un conjunto de mejores prácticas para trabajar en equipo y obtener el mejor resultado posible de un proyecto. Estas prácticas se apoyan unas a otras y su selección tiene origen en un estudio de la manera de trabajar de equipos altamente productivos.

SCRUM es una metodología propuesta por los japoneses Hirotaka Takeuchi e Ikujijo Nonaka, un modo de desarrollo de carácter adaptable y orientada a las personas antes
que a los procesos, que controla de forma empírica y adaptable la evolución del proyecto con las siguientes prácticas de la gestión ágil:
       - Revisión de las iteraciones.
       - Desarrollo incremental, en el proyecto no se trabaja con diseño o
        abstracciones.
       - Desarrollo evolutivo, intenta predecir en las fases iniciales cómo será el
         resultado final y sobre dicha predicción realizar el diseño, la estructura del
        - producto no es realista, porque las circunstancias obligaran a remodelarlo
          muchas veces.
       - Auto-organización, los equipos son auto-organizados, con margen de decisión
         suficiente para tomar las decisiones que consideren oportunas.
       - Y colaboración entre todos según su conocimiento y no según su rol o puesto.

El resultado final en esta metodología se consigue de forma iterativa e incremental. Al comienzo de cada iteración (sprint) se determina qué partes se van a construir,
tomando como criterios la prioridad para el negocio, y la cantidad de trabajo que se podrá abordar durante la iteración. Dichas iteraciones se presentan en las etapas de
Especulación, Exploración y Revisión; y debido a que, según este tipo de modelos de desarrollo nunca se termina un producto, se presentan de forma infinita pudiendo
llegar a la etapa de cierre solo cuando se desee enviar al mercado una versión funcional del producto.

Reglas básicas
SCRUM tiene un conjunto de reglas muy pequeño y muy simple y está basado en los principios de inspección continua, adaptación, auto-gestión e innovación.
El cliente se entusiasma y se compromete con el proyecto dado que ve crecer el producto iteración a iteración y encuentra las herramientas para alinear el desarrollo
con los objetivos de negocio de su empresa.
Por otro lado, los desarrolladores encuentran un ámbito propicio para desarrollar sus capacidades profesionales y esto resulta en un incremento en la motivación de los
integrantes del equipo.
Ya hemos dicho que SCRUM es fácil y sencillo pero vamos a la materia, los siguientes son los elementos básicos de SCRUM:
    
    •   Una lista con las funcionalidades de la aplicación ordenadas de mayor a menor
        importancia. Esta lista se llama "Product Backlog". No hace falta que esta lista
        contenga todas las funcionalidades inicialmente.
    •   De la lista anterior, se toman las primeras funcionalidades, se descomponen en
        tareas y son anotadas en una lista que se llama "Sprint Backlog". Estas tareas
        serán realizadas en el siguiente mes.

Además de estos elementos tenemos unas cuantas reglas básicas y sencillas que
tenemos que cumplir:
    •   Una vez que se pasan las tareas más prioritarias del "Product Backlog" al
        "Sprint Backlog", estas no se pueden cambiar, esto quiere decir, que el trabajo
        de un mes queda fijado. Esta es la regla más importante de todas.
    •   Al final del mes, periodo denominado "Sprint", se tiene que tener un ejecutable
        con las funcionalidades del "Sprint Backlog".
    •   Todo el equipo puede añadir funcionalidades al "Product Backlog", pero sólo
        una persona puede ordenarlo. A esta persona se le denomina "Product
        Owner". Es el responsable del producto final.
    •   Cada día se hace una reunión de menos de 15 minutos, en la que se reúne
        todo el equipo: ingenieros y gestor (llamado "SCRUM Master") en la que cada
        miembro del equipo expone sólo los siguientes temas:
             o ¿Qué es lo que se hizo el día anterior?
             o ¿Qué es lo que se va a hacer hoy?
             o ¿Qué impedimentos tengo para realizar mi trabajo?


        Sólo se tratan estos temas para que la reunión sea rápida y no malgastar el
        tiempo de los demás. Si se tiene que tratar otro tema se hace otra reunión sólo
        con las personas implicadas. ¿Recordáis la serie de TV "Canción triste de Hill
        Street" en la que el sargento tenía una reunión matutina con sus agentes y que
        terminaba con un "Tengan cuidado ahí fuera"? pues viene a ser algo parecido.
    •   Al final del mes, es decir, al final del Sprint, se presenta el producto y se toman
        del "Product Backlog" las funcionalidades para cubrir en el siguiente mes.

La calidad en este caso, se logra y se mantiene de forma continua, pues se involucra
al cliente durante todo el tiempo que tarde el desarrollo, permitiéndole hacer aportes
que enriquezcan y generen nuevas funcionalidades y/o características al producto que
se está desarrollando.
Básicamente esto es todo.
Como se puede observar las reglas son sencillas, claras y es muy fácil de explicar y de
entender, lo que ayuda mucho a su implantación, pero no hay que engañarse, SCRUM
es un proceso de cambio, y uno además bastante serio.
SCRUM es sencillo, pero duro como una piedra, y se encuentra siempre mucha
resistencia:
    •   Los jefes de proyecto no querrán comenzar los proyectos sin tenerlo todo
        perfectamente identificado, acotado y planificado.
    •   Los desarrolladores no querrán la               responsabilidad    de  estimar   las
        funcionalidades y demostrar el producto.
    •   Los gerentes no querrán dejar tranquilo al equipo durante los Sprints.
    •   Etc.

\section{DESCRIPCION DEL PROBLEMA}

%\end{document}


\section{Antecedentes}
Los sistemas de control de versiones en la actualidad son utilizados mayormente en la industria informática. Estos también son aplicados a otros ámbitos como documentos, imágenes, sitios web, etc.

Aunque un sistema de control de versiones puede realizarse de forma manual, es muy aconsejable disponer de herramientas que faciliten esta gestión, entre las cuales tenemos (CVS, Subversion, Mercurial, Git, etc.). Todos estos sistemas de control de versiones se basan en disponer de un repositorio, que es el conjunto de información gestionada por el sistema.

Git es un software de sistema de control de versiones diseñado por Linus Torvalds, pensando en la eficiencia y confiabilidad de mantenimiento de versiones de aplicaciones con una enorme cantidad de archivos de código fuente.

El diseño de Git se basó en BitKeeper y en Monotone. En principio, Git se pensó como un motor de bajo nivel que otros pudieran usar para escribir front end como Cogito o StGIT. Sin embargo, Git se ha convertido desde entonces un sistema de control de versiones con funcionalidad plena. Hay algunos proyectos de mucha relevancia que ya usan Git, en particular, el grupo de programación del núcleo del sistema operativo Linux.

\section{Definición del problema}
¿ Es posible la reducción de conflictos de inconsistencias debido a las modificaciones de ficheros por varias personas no coordinadas,  causadas por falta de comunicación de las personas del grupo de trabajo en el desarrollo de software que emplean el sistema de control de versiones distribuidas con GIT ?

\section{GIT}
\subsection{Modelo de objetos GIT}
Toda la información necesitada para representar la historia de un proyecto es almacenada en archivos referenciados por 40 dígitos.también denominado ``Object Name''
\subsection{Tipos de objetos}
\begin{itemize}
 \item \textbf{Blob} Usado para almacenar datos de archivos, generalmente archivos.
 \item \textbf{Tree} Básicamente igual a un directorio, hace referencia a otras ramas, arboles o blobs.
 \item \textbf{Commit} Apunta a un árbol simple, timestamp, autor de los cambios del anterior commit realizado.
 \item \textbf{Tag} Forma especifica para marcar un Commit. como por ejemplo una versión ralease especifico.
\end{itemize}


\section{Innovación Tecnológica}
En la actualidad la mayoría de los sistemas microblog mas usados, son de uso genérico utilizados para compartir información acerca del estado de acciones realizadas por los usuarios. Se realizara sistema web microblog que aporte a la comunicación enfocada a los integrantes de grupos de desarrollos de software, el cual permitirá realizar un mayor control en la sincronización en los repositorios subversionados con GIT.

\section{Cronograma}
%adicionamos el cronograma
\begin{center}
\begin{longtable}{|p{2cm}|p{2cm}|p{2cm}|p{2cm}|p{2cm}|p{1.3cm}|}
\hline
Objetivo General & Objetivo Específico & Actividad & Método & Resultado & Tiempo\\
\hline
\multirow{13}{2cm}{Desarrollar un sistema web microblog para notificación de acciones aplicadas en repositorios locales de desarrollo de proyectos de software versionados con GIT.} & 

\multirow{2}{2cm}{Elegir herramientas que aporten al desarrollo del software.} & {Investigar marcos de trabajo para la facilidad de elaboración de sistemas Web.} & \multirow{2}{2cm}{Búsqueda bibliográficas, entrevistas.} &\multirow{2}{2cm}{Se determinará las herramientas a ser utilizadas durante todo el transcurso del desarrollo del proyecto.}& \multirow{2}{2cm}{2}\\\cline{3-3}
& & {Investigar Herramientas CASE para apoyo al desarrollo del proyecto.}&&&\\\cline{2-6}

& \multirow{3}{2cm}{Analizar herramientas que faciliten la integración del sistema de control GIT con aplicaciones Web.}&{Hacer un análisis de ventajas y desventajas referentes a las opciones encontradas para el desarrollo}&\multirow{3}{2cm}{Busquedas bibliográficas, Entrevistas}&\multirow{3}{2cm}{Se habrán realizado pruebas para la factibilidad de la comunicación entre las maquinas locales(Cliente) y el sistema web.
Se tendrá montado el ambiente de trabajo para la realización del proyecto.}&\multirow{3}{2cm}{2}\\\cline{3-3}
& &{Realizar la instalación del ambiente de trabajo.}&&&\\\cline{3-3}
& &{Realizar pruebas de comunicación con una maquina local a un sistema web simple}&&&\\\cline{2-6}
& \multirow{3}{2cm}{Realizar un asistente de configuración, para la comunicación de repositorios con el sistema Web.}&{Hacer una revisión de código de herramientas ya desarrolladas previamente.}&\multirow{3}{2cm}{Búsqueda bibliográfica. Analisis de código de programas ya realizados previamente.}&\multirow{3}{2cm}{Se determinará la forma de configuración, la posibilidades de fallo presentadas, se elaborarán los scripts para la comunicación cliente-sistema web.}&\multirow{3}{2cm}{2}\\\cline{3-3}
& &{Realizar scripts que faciliten la configuración de la comunicación cliente-sistema Web.}&&&\\\cline{3-3}
& &{Realizar pruebas de configuración.}&&&\\\cline{2-6}

&\multirow{6}{2cm}{Desarrollar el sistema web microblog, que permita registrar las acciones realizadas(commit) por los desarrolladores en los repositorios locales.}&{- Inicio del Scrpint, Realizar una lista de requerimientos para el desarrollo del sistema según iteracion.}&\multirow{6}{2cm}{Busqueda bibliográfica, entrevistas, Respetando normas de la metodología SCRUM, Se realizan scripts de pruebas(unitTest y clientTest)}&\multirow{6}{2cm}{Se elaborarán los prototipos según cada iteración, esta comprende un total de 4 iteraciones.}&\multirow{2}{2cm}{2}\\\cline{3-3}

& &{Realizar algunos diagramas UML de apoyo para el desarrollo del sistema}&&&\\\cline{3-3}
& &{Determinación del alcance de los objetivos a cumplir durante la iteración}&&&\\\cline{3-3}
& &{Inicio del desarrollo del sistema segun iteración}&&&\\\cline{3-3}
& &{Pruebas de validación al prototipo}&&&\\\cline{3-3}
& &{Retrospectiva de las actividades realizadas en el scprint actual}&&&\\
\hline
\end{longtable}
\end{center}

%adicionamos la bibliografia
\begin{thebibliography}{2007}
\bibitem{}{LOELIGER, JON.Version control with Git, Ed. O'REILLY, 2009}
\bib
\bibitem{Orelly}Learning Python, 3a edition.
\bibitem{Wrox}Professional Python Framework Web 2.0 Programing with Django and Turbogears,2007.
\bibitem{Springer}Python Scripting for Computational Science, 3rd. edititon(2008). 
\bibitem{Pressman}Ingenieria de Sfotware,Roger Pressman
\bibitem{Scrum}Metodología agil SCRUM, http://www.proyectosagiles.org/que-es-scrum
\end{thebibliography}
\end{document}
%Cerramos el bloque para escribir el documento.
