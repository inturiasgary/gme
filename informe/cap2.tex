%\documentclass[12pt]{article}
%\usepackage[latin1]{inputenc}
%\usepackage[spanish]{babel}
%\usepackage{latexsym}
%\usepackage{amssymb}
%\usepackage{amsmath}

%\setlength{\textwidth}{15cm}

%\newtheorem{ejem}{Ejemplo}
%\newtheorem{teor}{Teorema}
%\begin{document}

\chapter{MARCO TEORICO}
\section{DEFINICIONES BÁSICAS}
\subsection{EL PROCESO DE DESARROLLO DE SOFTWARE}
En una actividad en el cual el proceso es inherentemente social y creativo,  y como tal tenemos que hacernos de la idea de que la mayoría de los problemas que acarrea son de esta índole.
En aspectos generales sin enfocarnos al concepto de cada metodología utilizada para realizar esta labor, podemos ver que nos enfocamos en la evolución del código.

Ya sea que trabajemos en grupo o de forma individual, esta actividad se comienza con una hoja en blanco y una idea en la cabeza, que a medida que progresamos, va tomando forma.
Enfocandonos desde el punto de vista del código realizado durante este proceso, al principio arrancamos con un directorio vació, en el cual vamos escribiendo, construyendo nuestro software
de forma incremental e iterativa, corrigiendo nuestros propios errores a medida que aparecen, y agregando cosas nuevas cuando se nos place.

Nuestro código va evolucionando, cambiando con iteraciones pequeñas que nosotros vamos introduciendo por diversos motivos como, correcciones o características nuevas en el software.
Estos cambios realizados en el código son de manera organizada y ordenada, debido a que corresponden a alguna construcción mental nuestra, porque cuando nosotros pensamos en implementar algo o corregir un bug, lo tenemos en mente como una unidad, como un objetivo bastante independiente de como se representa en el código fuente; y nosotros 
pensamos en que cambios le tenemos que realizar para lograr ese objetivo propuesto.

\subsection{GRUPOS DE TRABAJO EN EL DESARROLLO DE SOFTWARE}
En la actividad de desarrollo de software con frecuencia los desarrolladores son organizados en grupo, independientemente de que metodología empleen. Surgiendo la suma necesidad de coordinar el trabajo, no solo por una cuestión natural sino también como mecanismo para optimizar los recursos.
Por eso, para desarrollar en grupo es imprescindible la buena comunicación y el entendimiento entre los pares, Esto implica que, en general, los desarrollos se dan de forma coordinada (ya sea de manera horizontal o vertical, independientemente del mecanismo que se elija explícita o implícitamente para ello) al menos en un nivel social, las tareas se reparten y los cambios se discuten donde afectan al grupo para facilitar el trabajo.

Claro que el desarrollar en grupo, por más que uno se lleve maravillosamente bien con la gente involucrada, acarrea ciertas incomodidades que sí son más técnicas. Al haber más de una persona modificando el código fuente de forma simultanea, existe una complejidad, y nada menor, en lo que refiere al hecho de sincronizarlo y mantenerlo coherente entre todos los miembros del grupo.

También se dan en un grupo de trabajo relaciones asimétricas respecto del código, debido a que cada grupo tiene una forma y un flujo de trabajo particular, en el cual, por ejemplo, se pueden dar relaciones jerárquicas, revisión de código entre pares, subgrupos, etc.

Esto va a reflejarse en el código fuente con el surgimiento de una nueva necesidad, que va a ser la de la integración de múltiples trabajos individuales, la distribución del mismo en distintas maquinas y la coordinación para que todos puedan trabajar sobre la misma base de código.

\subsection{SISTEMA DE CONTROL DE VERSIONES}

El desarrollo de proyecto de software en la actualidad se encuentra a cargo de un equipo de desarrolladores, como así también los administradores del proyecto.

La comunicación entre los integrantes es esencial para el cumplimiento de los objetivos a alcanzar, para la coordinación eficaz del equipo de trabajo y la determinación de tiempos para las respectivas tareas designadas.

La dificultad de tratar con cada uno de los integrantes de forma personal aumenta según el numero que estos integran y aun mas si realizan sus trabajos a distancia, esto causa mala comprensión de las tareas a cumplir dentro del desarrollo de software. Provocando perdidas de tiempo que en el transcurso son esenciales para la culminación del proyecto.

La mayoría de sistemas de control de versiones no se encuentran integrados a un sistema microblog, el cual facilitaría las notificaciones de las acciones que efectúan en tiempo real a los ficheros de los repositorios de trabajo. Así provocando una de-sincronización causando conflictos a la hora de realizar actualizaciones en los repositorios.

GIT no es una excepción a estos posibles problemas, debido a que cada integrante del grupo de desarrollo obtiene una copia idéntica del repositorio, es una de las cualidades de los sistemas de control de versiones distribuidas.
\subsubsection{CLASIFICACIÓN}

\begin{itemize}
\item \textbf{CENTRALIZADOS :}Se basan en un repositorio único central con toda la información de cambios realizados el cual es accesible por todos los desarrolladores.\\
al haber un solo repositorio, este es el encargado del manejo de las ramas, quedando todos dentro de este. Están basados en una linea de tiempo, necesitan un servidor con funcionamiento de tiempo completo y con conexión permanente
configurado con algún sistema de autentificacion y permisos, por tal motivo la configuración suele ser mas compleja, según la necesidad. 
una de las desventajas mayores que representa es la dependencia de conexión con el servidor, ya que toda la información es manejada por el repositorio. 
Generalmente hay dos tipos el Lock-Modify-Unlock y los que usan el modelo Copy-Modify-Merge. el primero es el mas simple y limitado, y consiste en que cada vez que un usuario quiere editar un archivo, este se bloquea y por lo tanto no puede ser
modificado por nadie mas, hasta que el usuario lo desbloquee. El segundo modelo propone lo siguiente: se hace una copia del estado actual del repositorio, se modifica y se aplica el conjunto de cambios al repositorio central haciendo un merge. El cual es una forma de trabajo mas conveniente.
\item \textbf{DISTRIBUIDOS :}Carece de un punto central de desarrollo por tanto los repositorios están distribuidos y descentralizados en diferentes maquinas que pueden o no ser independientes entre si, y 
técnicamente no hay ninguno mas importante que otro. Esto trae bastante beneficios al desarrollar, dado que cada desarrollador tenga su repositorio propio sobre el cual trabaje de una forma independiente, y periódicamente se pongan en común los trabajos de todos en algún repositorio convenido a tal efecto.

\end{itemize}

\subsection{MICROBLOG}

\subsection{FRAMEWORK DJANGO}
Desarrollar un sistema Web microblog\footnote{Microblog es un servicio que permite a sus usuarios enviar y publicar mensajes breves.
} para notificación de acciones aplicadas en repositorios locales de desarrollo de proyectos de software versionados con GIT\footnote{GIT software de sistema de control de versiones distribuido.
}.

\subsection{WEB API}
Una API es una interfaz de programación de aplicaciones (del inglés API: Application Programming Interface). Es un conjunto de rutinas que provee acceso a funciones de un determinado software.

Son publicadas por los constructores de software para permitir acceso a características de bajo nivel o propietarias, detallando solamente la forma en que cada rutina debe ser llevada a cabo y la funcionalidad que brinda, sin otorgar información a cerca de como se lleva a cabo la tarea. Son utilizadas por los programadores para construir sus aplicaciones sin necesidad de volver a programar funciones ya hechas por otros, reutilizando código que se sabe que está probado y que funciona correctamente.

En la web las API's tambien son sinonimos de Web Service, las API's son publicadas por sitios para brindar la posibilidad de realizar alguna acción o acceder a alguna característica o contenido que el sitio provee. 

Una API puede ser:

\begin{itemize}
 \item \textbf{DEPENDIENTE DEL LENGUAJE:} Que es disponible solo por un lengiaje de programacion dado, usando la sintaxis y elementos de ese lenguaje para hacer conveniente el uso del API en esa peticion.
 \item \textbf{INDEPENDIENTE DEL LENGUAJE:} Que puede ser llamado por un gran numero de de lenguajes de programacion. Esta es una característica de un servicio al estilo de la API que no está vinculada a un determinado proceso o sistema y está disponible como una llamada a procedimiento remoto(RPC Remote ).
\end{itemize}

\subsubsection{XML RPC}
Es un protocolo de llamada a procedimiento remoto, que utiliza HTTP como el transporte y XML para la codificacion de datos. XML-RPC esta diseñado para ser lo mas sencillo posible, permitiendo al mismo tiempo las estructuras mas complejas de datos que deben ser transmitidos, procesados y devueltos.

XML-RPC contra. Otros protocolos

XML-RPC no es la única manera de hacer llamadas de procedimiento remotos. Otros protocolos populares incluyen CORBA, DCOM y el SOAP. Cada uno de estos protocolos tiene ventajas y desventajas.

XML-RPC contra CORBA

CORBA es un protocolo popular para la escritura distribuida, usos orientados a objetos. Se utiliza típicamente en usos de varias filas de empresas. Recientemente, también es adoptado por el proyecto del gnome para la comunicación de interaplicacion.

CORBA es muy apoyado por muchos vendedores y varios proyectos de software libre. CORBA trabaja bien con Java y C++, y está disponible para muchos otras idiomas. CORBA también proporciona una lengua excelente de la definición de interfaz (IDL), permitiendo que usted defina APIs legibles, orientados al objeto.

Desafortunadamente, CORBA es muy complejo. Tiene una curva de aprendizaje escarpada, requiere esfuerzo significativo para ejecutar, y requiere a clientes bastante sofisticados.

XML-RPC contra DCOM

DCOM es respuesta de Microsoft a CORBA. Es grande si usted está utilizando ya componentes de COM, y usted no necesita hablar con los sistemas que no son de Microsoft. entonces, no le ayudará mucho.

XML-RPC contra SOAP

SOAP es muy similar a XML-RPC. , Trabaja también formando llamadas de procedimiento sobre el protocolo HTTP con documentos de XML. Desafortunadamente, el SOAP parece sufrir de arrastramiento de la especificación.

SOAP fue creado originalmente como colaboración entre UserLand, DevelopMentor y Microsoft. El lanzamiento público inicial era básicamente XML-RPC con los namespaces \footnote{En programación, un espacio de nombres (del inglés namespace), en su acepción más simple, es un conjunto de nombres en el cual todos los nombres son únicos.} y nombres de elemento más largos. Desde entonces, sin embargo, SOAP se ha volcado un grupo de trabajo de W3C.

Desafortunadamente, el grupo de trabajo ha estado agregando una lista de características extrañas al SOAP. En fecha la escritura actual, el SOAP apoya esquemas de XML, enumeraciones, híbridos extraños de structs y de órdenes, y tipos de encargo. Al mismo tiempo, varios aspectos SOAP son puesta en práctica definida.

%\end{document}
