% Todas las disertaciones deber�an contener una introducci�n. Todav�a no ha sido realizada, % pero la introducci�n deber�a contener los conceptos, background, y el objetivo de la disertacion.

% Autor : Patricia E. Romero Rodriguez
% Fecha de ultima modificacion: 09 - agosto - 2007

%\documentclass[12pt]{article}
%\usepackage[latin1]{inputenc}
%\usepackage[spanish]{babel}
%\usepackage{latexsym}
%\usepackage{amssymb}
%\usepackage{amsmath}

%\setlength{\textwidth}{15cm}

%\newtheorem{ejem}{Ejemplo}
%\newtheorem{teor}{Teorema}
%\begin{document}
\chapter{Presentación}
\section{Introducción}
El instituto Normal manuel asencio villarroel de parac
\section{Planteamiento del problema}
El desarrollo de proyecto de software en la actualidad se encuentra a cargo de un equipo de desarrolladores, como así también los administradores del proyecto.

La comunicación entre los integrantes es esencial para el cumplimiento de los objetivos a alcanzar, para la coordinación eficaz del equipo de trabajo y la determinación de tiempos para las respectivas tareas designadas.

La dificultad de tratar con cada uno de los integrantes de forma personal aumenta según el numero que estos integran y aun mas si realizan sus trabajos a distancia, esto causa mala comprensión de las tareas a cumplir dentro del desarrollo de software. Provocando perdidas de tiempo que en el transcurso son esenciales para la culminación del proyecto.

La mayoría de sistemas de control de versiones no se encuentran integrados a un sistema microblog, el cual facilitaría las notificaciones de las acciones que efectúan en tiempo real a los ficheros de los repositorios de trabajo. Así provocando una de-sincronización causando conflictos a la hora de realizar actualizaciones en los repositorios.

GIT no es una excepción a estos posibles problemas, debido a que cada integrante del grupo de desarrollo obtiene una copia idéntica del repositorio, es una de las cualidades de los sistemas de control de versiones distribuidas.
\section{Objetivos}
\subsection{Objetivo General}
Desarrollar un sistema Web microblog\footnote{Microblog es un servicio que permite a sus usuarios enviar y publicar mensajes breves.
} para notificación de acciones aplicadas en repositorios locales de desarrollo de proyectos de software versionados con GIT\footnote{GIT software de sistema de control de versiones distribuido.
}.

\subsection{Objetivos Específicos}
\begin{itemize}
\item Elegir herramientas que aporten al desarrollo del sistema
\item Analizar herramientas que faciliten la integración del sistema de control 	de versiones GIT con aplicaciones Web.
\item Realizar un asistente de configuración, para la comunicación de 	repositorios con el sistema Web(lado cliente).
\item Desarrollar el sistema web microblog, que permita registrar las acciones realizadas(commit) por los desarrolladores en los repositorios locales(lado servidor).
\end{itemize}

\section{Justificación}
El sistema web que se desea desarrollar ayudara  a los integrantes del grupo de trabajo encargados en la realización del proyecto, facilitando hacer un seguimiento de las tareas a realizar, el reporte de avances que se realizan durante el transcurso del día, así también poder compartir información y reportar errores entre el grupo.
Los proyectos gestionados con el sistema de control de versiones GIT se podrán integrar con el sistema web microblog, debido a que se hará uso de algunos ganchos(hooks) que enlazarán con el sistema y así poder notificar a los integrantes del grupo de desarrollo en tiempo real sobre las acciones(commit) realizadas en cada uno de sus repositorios locales.
\section{Metodología}
SCRUM es una metodología ágiles, toma en cuenta la situación cambiante de los requisitos del cliente, así también minimiza la necesidad de la documentación.
Organizando el desarrollo de software de un manera evolutiva en los denominados sprints(periodos que duran de 15 a 30 días).


%\end{document}
