%\documentclass[12pt]{article}
%\usepackage[latin1]{inputenc}
%\usepackage[spanish]{babel}
%\usepackage{latexsym}
%\usepackage{amssymb}
%\usepackage{amsmath}

%\setlength{\textwidth}{15cm}

%\newtheorem{ejem}{Ejemplo}
%\newtheorem{teor}{Teorema}
%\begin{document}
\chapter{HERRAMIENTAS PARA EL DESARROLLO DE SOFTWARE}
\section{INTRODUCCION}
\section{HERRAMIENTAS COLABORATIVAS}
\subsection{GIT}
\subsubsection{HOOKS(GANCHOS)}
No son copiados durante la ejecucion del comando \textit{git clone}.En otras palabras, los hooks configurados en nuestro repositorio privado no son propagados ni alteran el 
comportamiento del nuevo clon.
La clasificacion de los hooks son:
\begin{itemize}
 \item \textit{"pre" Hooks} se ejecuta antes de que la accion se complete. Este tipo de hooks es mayormente usado para \textbf{falta del libro}
 \item \textit{"post" Hooks} se ejecuta despues de que la accion se complete y puede ser usado para noficaciones o ejecutar un proceso adicional.
\end{itemize}

\textbf{poner la imagen de la secuencia de git commit}

%\end{document}
