\begin{titlepage}

\begin{center}
 {\Large FICHA RESUMEN}\\
\end{center}

El desarrollo de proyecto de software en la actualidad se encuentra a cargo de un equipo de desarrolladores, como así también los administradores del proyecto.
La comunicación entre los integrantes es esencial para el cumplimiento de los objetivos a alcanzar, para la coordinación eficaz del equipo de trabajo y la determinación de tiempos para las respectivas tareas designadas.\\

La dificultad de tratar con cada uno de los integrantes de forma personal aumenta según el numero que estos integran y aun mas si realizan sus trabajos a distancia, esto causa mala comprensión de las tareas a cumplir dentro del desarrollo de software. Provocando perdidas de tiempo que en el transcurso son esenciales para la culminación del proyecto.\\

La mayoría de sistemas de control de versiones no se encuentran integrados a un sistema que de a conocer las actividades que se realizan en cada uno de los repositorios que modifican cada integrante desarrollador, Así provocando una sincronización causando conflictos a la hora de realizar actualizaciones en los repositorios.\\
El sistema de control de versiones GIT no es una excepción a estos posibles problemas, debido a que cada integrante del grupo de desarrollo obtiene una copia idéntica del repositorio, el cual es una de las cualidades de los sistemas de control de versiones distribuidas.\\

Es por esto que surge la idea de aportar con una herramienta para facilitar las notificaciones de las acciones que se efectúan en los repositorios de trabajo, mediante un sistema web microblog y así poder reducir los conflictos de inconsistencias debido a las modificaciones de ficheros por varias personas no coordinadas, también se podrá registrar las listas de tareas asignadas a cada integrante 
perteneciente al repositorio en desarrollo.


\end{titlepage}